\section{Introduzione}

La crittografia è una scienza antichissima che consiste nel codificare e
decodificare l'informazione.
L'operazione di codifica permette di ottenere un testo codificato a partire da
un “testo in chiaro” che
può essere letto da tutti.
L'operazione di decodifica invece, consiste nel ricavare un testo in chiaro
partendo da un
testo cifrato. Ambedue le operazioni si basano su un algoritmo e sulla chiave;
l'algoritmo è
certamente pubblico. La sicurezza del sistema è data dalla segretezza della
chiave e dalla
robustezza dell'algoritmo. Esistono due tecniche di crittografia:
la crittografia \textit{simmetrica} e la
crittografia \textit{asimmetrica}, quest'ultima più recente.
