\section{Funzioni Hash}

Una funzione di hash è una funzione matematica che permette di ridurre una
qualunque stringa di
testo (indipendentemente dalla sua lunghezza) in una nuova stringa avente
precise caratteristiche
tra cui un numero di caratteri predefinito. A partire da un input X, in
altre parole, sarà possibile
generare un input Y che avrà delle caratteristiche ben definite.
A prescindere dal tipo di funzione di hashing tutte hanno dei punti in comune:

\begin{itemize}
    \item \textbf{Costanza}: a parità di input la stessa funzione di hash
          restituirà sempre la stessa stringa
          alfanumerica; ad ogni input, in altre parole, corrisponde sempre e
          inevitabilmente lo stesso
          output;
    \item \textbf{Irreversibilità}: mentre è sempre possibile riprodurre
          l'output conoscendo l'input originario
          col quale l'output è stato generato non è però possibile fare
          il percorso inverso, non si può
          quindi a partire da una stringa alfanumerica risalire al contenuto
          iniziale dell'input;
    \item \textbf{Determinismo}: indipendentemente da quanto sia lungo l'input,
          la funzione di hash restituirà
          sempre una stringa alfanumerica di un numero determinato di caratteri;
    \item \textbf{Effetto} valanga: non importa quanto l'input sia lungo e
          complesso, è sufficiente una
          variazione infinitesimale dell'input per generare un output
          completamente differente;
\end{itemize}