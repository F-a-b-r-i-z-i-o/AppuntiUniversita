\chapter{SQL Injection}

SQL injection è una tecnica di \textit{code injection}, usata per attaccare
applicazioni di gestione
dati, con la quale vengono inserite delle stringhe di codice SQL malevole
all'interno di campi
di input in modo che queste ultime vengano poi eseguite
(ad esempio per fare inviare il contenuto del database all'attaccante).
Sfrutta quindi le vulnerabilità di sicurezza del codice
di un'applicazione. Un classico esempio è un'applicazione che si interfaccia
con un database
e che prende in input dati provenienti da una form web.
SQL Injection è comune con PHP e ASP (ormai abbandonato) a causa della
prevalenza di
vecchie interfacce funzionali.
Per quanto sia estremamente semplice evitare questo tipo di problemi,
molte applicazioni
commerciali (e non) sono soggette a questa vulnerabilità
(che può portare ad accessi non
autorizzati e alla distruzione di database aziendali) dovuta all'eccessiva
fiducia negli input
degli utenti.\\

Potenziali vulnerabilità ad SQL injection si verificano quando:

\begin{itemize}
    \item I dati entrano in un programma da una fonte non attendibile.
    \item I dati in input sono utilizzati per costruire
          dinamicamente una query SQL.
\end{itemize}

Le principali conseguenze di un attacco SQL Injection sono:

\begin{itemize}
    \item \textbf{Riservatezza} (confidentiality):
          Poiché i database SQL contengono generalmente dati
          sensibili, la perdita di riservatezza è un problema frequente con le
          vulnerabilità SQL
          Injection;
    \item \textbf{Autenticazione} (authentication): Se si usano comandi SQL
          obsoleti e non sicuri per controllare i
          nomi utente e le password, può essere possibile connettersi ad un sistema
          come un
          altro utente senza alcuna conoscenza della password;
    \item \textbf{Autorizzazione} (authorization): Se le informazioni di
          autorizzazione sono conservate
          in un database SQL, può essere possibile modificare queste informazioni
          attraverso
          lo sfruttamento di una vulnerabilità SQL Injection;
    \item \textbf{Integrità} (integrity): Così come è possibile leggere
          informazioni sensibili, è anche
          possibile apportare modifiche o addirittura cancellare queste informazioni
          con un
          attacco SQL Injection
\end{itemize}

In sostanza, l'attacco viene effettuato inserendo un metacarattere all'interno
dei dati in input, per poi
posizionare i comandi SQL nel piano di controllo che prima non esisteva in quel
piano.
Questo difetto dipende dal fatto che l'SQL non fa una reale distinzione tra il
piano di controllo
e quello dei dati.

\section{Esempi}

\subsection{Esempio 1}

Prendiamo in esame il seguente codice SQL:

\begin{lstlisting}[language=Sql]
    SELECT id, firstname, lastname
    FROM authors
    WHERE firstname = $input1 AND lastname = $input2
\end{lstlisting}

Assumiamo che il valore delle due variabili \verb|\$input1| \verb|\$input2|
sia:

\begin{lstlisting}[language=PHP]
    $input1 = "evil'ex"
    $input2 = "Newman"
\end{lstlisting}

La query quindi risulterà essere la seguente:

\begin{lstlisting}[language=Sql]
    SELECT id, firstname, lastname
    FROM authors
    WHERE firstname = 'evil'ex' AND lastname = 'Newman'
\end{lstlisting}

Avremo quindi un errore, in quanto il database tenterà di eseguire \verb|evil|
come un comando SQL (\verb|Incorrect syntax near il'|).

\subsection{Esempio 2}

Prendiamo in esame il seguente codice \verb|C#|:

\begin{lstlisting}[language=C++]
    string userName = ctx.getAuthenticatedUserName();
    string query = 
        "SELECT * FROM items WHERE owner = '" + userName + 
        "' AND itemname = '" + ItemName.Text + "'";
    sda = new SqlDataAdapter(query, conn);
    DataTable dt = new DataTable();
    sda.Fill(dt);
    ...
\end{lstlisting}

Se l'utente \verb|wiley| imposta il valore dell'attributo\\
\verb|ItemName.Text| a \verb|name' OR 'a'='a'|, la query risulterà essere:

\begin{lstlisting}[language=SQL]
    SELECT * FROM items
    WHERE (owner = 'wiley' AND itemname = 'name')
        OR 'a'='a';
\end{lstlisting}

Che il database interpreterà come:

\begin{lstlisting}[language=SQL]
    SELECT * FROM items;
\end{lstlisting}

\subsection{Esempio 3}

Facciamo sempre riferimento al codice dell'esempio precedente.
Mettiamo che l'utente \verb|hacker| inserisca come input la stringa:

\begin{lstlisting}[language=SQL]
    "name'); DELETE FROM items; --"
\end{lstlisting}

La query risulterà essere:

\begin{lstlisting}[language=SQL]
    SELECT * FROM items
    WHERE owner = 'hacker'
        AND itemname = 'name';

    DELETE FROM items;
    --'
\end{lstlisting}

Siamo quindi riusciti ad eliminare l'intera tabella !\\
Questo accade perché molti database (incluso Microsoft SQL Server 2000)
permettono l'esecuzione consecutiva di più query separata da un \verb|;|.
Per esempio, questo non funzionerebbe in Oracle DB in quanto non permette questo
comportamento.

\section{Mitigation}

In generale quando si ha a che fare con un database e con query che richiedono dati
provenienti dall'esterno è sempre consigliato adottare strategie di mitigazione.
Un approccio tradizionale per prevenire gli attacchi SQL injection è quello di
gestirli come un
problema di validazione degli input e poi:

\begin{itemize}
    \item o accettare solo i caratteri di una \textbf{whitelist} di valori sicuri.
          Tutto il resto è quindi negato;
    \item o identificare e rendere sicuri (\textit{escaping}) una serie di valori
          presenti all'interno di una \textbf{blacklist} di elementi
          potenzialmente dannosi. Quindi è
          tutto consentito tranne ciò che è specificato nella blacklist.
          Questa però va
          costantemente aggiornata.
\end{itemize}

Successivamente andremo ad elencare alcune delle tecniche di difesa più efficaci
contro SQL Injection.

\subsection{Dichiarazioni Preparate}

Tutti i linguaggi garantiscono la possibilità di “preparare” dei comandi.
L'uso di dichiarazioni preparate con vincoli variabili, anche dette
\textit{query parametrizzate}, è il
modo corretto con cui tutti gli sviluppatori dovrebbero prima essere istruiti su
come scrivere
le query del database.
Le query parametrizzate costringono lo sviluppatore a definire prima tutto il
codice SQL, per
poi passare ogni parametro alla query in secondo momento.
Le istruzioni preparate assicurano che un aggressore non sia in grado di
modificare l'intento
di una query, anche se vengono passati come input dei comandi SQL.\\

Alcuni esempi pratici:

\begin{itemize}
    \item \textbf{Java EE}:
          utilizzare  \verb|PreparedStatement()| con le bind variables;
    \item \textbf{.NET}:
          utilizzare query parametrizzate come \verb|SqlCommand()| o \verb|OleDbCommand()|
          con le bind variables;
    \item \textbf{PHP}: usare i \verb|PDO| con query parametrizzate fortemente tipizzate
          (\textit{strongly typed parameterized queries}) come \verb|bindParam()|
    \item \textbf{Hibernate}: usare \verb|createQuery()|
          con le bind variables (vengono chiamate \textit{named parameters}
          in Hibernate)
    \item \textbf{SQLite}: sfruttare \verb|sqlite3_prepare()|
          pre creare degli statement object.
\end{itemize}

\paragraph{Un esempio di codice Java:} \

\begin{lstlisting}[language=Java]
    // This should REALLY be validated too
    String custname = request.getParameter("customerName");
    String query = "
      SELECT account_balance 
      FROM user_data WHERE
      user_name = ? 
    ";

    // qui viene effettuato il binding e reso sicuro l'input
    PreparedStatement pstmt = connection.prepareStatement( query );
    pstmt.setString( 1, custname);

    ResultSet results = pstmt.executeQuery( );
\end{lstlisting}

\subsection{Procedura Memorizzata}

Il codice SQL per una \textbf{stored procedure} viene definito e memorizzato nel
database stesso e
poi richiamato dall'applicazione. Vanno sempre passati i parametri.

\begin{lstlisting}[language=Java, caption=Esempio di codice Java che
    sfrutta una \textit{stored procedure}]
    String custname = request.getParameter("customerName");
    try {
        CallableStatement cs = connection.prepareCall(
            "{callsp_getAccountBalance(?)}"
            );
        cs.setString(1, custname);
        ResultSet results = cs.executeQuery();
        // result set handling
    } catch (SQLException se) {
        // logging and error handling
    }
\end{lstlisting}

\subsection{Whitelist Input Validation}

I valori dei parametri devono essere mappati con i nomi di tabelle o di colonne
validi (o comunque previsti) per assicurarsi che l'input dell'utente non
convalidato non finisca nella
query. Devono effettivamente esistere nel database.

\begin{lstlisting}[language=Java, caption=Esempio di codic Java che effettua 
    input validation tramite whitelist.]
    String tableName;
    switch(PARAM):
        case "Value1": tableName = "fooTable";
            break;
        case "Value2": tableName = "barTable";
            break;
        ...
        default: 
            throw new InputValidationException("
            unexpected value provided for table name
            ");
\end{lstlisting}

\subsection{Escaping degli Input}

Questa tecnica dovrebbe essere usata solo come ultima risorsa,
quando nessuna delle
precedenti è fattibile. Può essere combinata però con le altre tecniche.
La Whitelist Input Validation è probabilmente la scelta migliore
in quanto questa
metodologia è fragile rispetto ad altre difese e non possiamo garantire che
impedirà tutte le
SQL injection in ogni situazione.
Ogni DBMS supporta uno o più caratteri di escape per determinati tipi di query.
Se si esegue
l'escape di tutti gli input forniti dall'utente utilizzando lo schema di escape
appropriato per il
database che si sta utilizzando, il DBMS non confonderà quell'input con il codice
SQL scritto
dallo sviluppatore, evitando così eventuali vulnerabilità di SQL injection.\\

In PHP generalmente si utilizza la funzione \verb|mysqli_real_escape_string()|
per fare escaping dei parametri da utilizzare nella query prima di mandarla
al database.

\begin{lstlisting}[language=Php]
    $mysqli = new mysqli(
        'hostname', 
        'db_username', 'db_password', 
        'db_name'
        );
    $query = sprintf(
        "SELECT * 
        FROM `Users` 
        WHERE UserName='%s' AND Password='%s'", 
        $mysqli->real_escape_string($username), 
        $mysqli->real_escape_string($password));
    $mysqli->query($query);
\end{lstlisting}

Questa funzione antepone il carattere \verb|\| (backslash) ai seguenti caratteri,
effettuando quindi operazioni di escaping: \verb|\\x00|, \verb|\\n|, \verb|\\r|,
\verb|\\|, \verb|\'|, \verb|\"| e \verb|\\x1a| (EOF).

\subsection{Altre Difese Aggiuntive}

Oltre ad adottare una delle quattro difese primarie (quelle elencate in precedenza),
è raccomandato anche di adottare tutte
queste tecniche per assicurare una difesa migliore.

\paragraph{Least Privilege.}
Per ridurre i potenziali danni di un attacco SQL injection, è necessario ridurre
al minimo i
privilegi assegnati ad ogni account. Non assegnate diritti di accesso di tipo DBA
o admin agli
account dell'applicazione. È meglio partire da zero per determinare quali diritti
di accesso
richiedono gli account piuttosto che cercare di capire quali diritti di accesso
si devono togliere.
In breve:

\begin{itemize}
    \item Assicuratevi che agli account che necessitano solo di un accesso in
          lettura sia
          concesso solo l'accesso in lettura alle tabelle a cui hanno bisogno di accedere;
    \item Se un account ha bisogno di accedere solo ad alcune porzioni di una
          tabella,
          considerare la possibilità di creare una vista che limiti l'accesso a quella
          porzione di
          dati e di assegnare invece l'accesso dell'account alla vista, piuttosto che
          alla tabella sottostante;
    \item Raramente, se non mai, concedere l'accesso totale agli account del
          database.
\end{itemize}

\paragraph{Multiple DB users.}
Diversi DB users possono essere utilizzati per diverse applicazioni web.
In questo modo, il progettista dell'applicazione può avere una buona granularità
nel controllo degli accessi e dunque,
ogni utente del DB avrà l'accesso selezionato solo a ciò di cui ha bisogno, e
l'accesso in scrittura a seconda delle necessità. Un esempio è la pagina di login
di un sito. Questa avrà sicuramente bisogno dei diritti di accesso in lettura alla
tabella contenente gli username e le password, ma NON dovrà poter scriverci in
nessun modo (non richiede quindi diritti di scrittura su questa tabella).

\paragraph{Views.}
È possibile utilizzare le viste SQL per aumentare ulteriormente la granularità
dell'accesso, limitando l'accesso in lettura a specifici campi di una tabella o
ai join di tabelle.

\paragraph{Input validation.}
Oltre ad essere una difesa primaria quando non è possibile nient'altro,
la validazione degli
input può anche essere una difesa secondaria utilizzata per rilevare gli input
non autorizzati
prima che questi vengano passati alla query SQL.